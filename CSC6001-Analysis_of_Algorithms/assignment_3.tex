\documentclass[11pt, a4paper, oneside]{memoir}

% 数学公式包
\usepackage{amsmath, amssymb, amsthm}
\usepackage{mathtools}

% 算法包
\usepackage{algorithm}
\usepackage{algorithmic}

% 代码高亮
\usepackage{listings}
\usepackage{xcolor}

% 图表包
\usepackage{graphicx}
\usepackage{tikz}
\usepackage{pgfplots}
\pgfplotsset{compat=1.18}

% 表格包
\usepackage{booktabs}
\usepackage{array}
\usepackage{multirow}

% 中文支持
% \usepackage[UTF8]{ctex}

% 其他实用包
\usepackage{hyperref}
\usepackage{geometry}
\usepackage{enumitem}
\usepackage{url}
\usepackage{float}
\usepackage{lipsum} % 随机文本包

% ===== 页面布局 =====
\setlrmarginsandblock{3cm}{3cm}{*} % 左边距3cm,右边距3cm
\setulmarginsandblock{2.5cm}{1.5cm}{*} % 上边距2.5cm,下边距1.5cm
\setlength{\beforechapskip}{0.5cm}
\checkandfixthelayout

% ===== 页面风格 =====
\makeevenfoot{headings}{}{\thepage}{}
\makeoddfoot{headings}{}{\thepage}{}
\makeevenhead{headings}{}{\leftmark}{}
\makeoddhead{headings}{}{\leftmark}{}
\makeheadrule{headings}{\textwidth}{0.4pt}

% ===== 章节格式 =====
\makeatletter
\renewcommand{\chaptername}{} % 移除“Chapter”字样
\renewcommand{\printchapternum}{} % 不打印章节号
\renewcommand{\afterchapternum}{} % 移除章节号后的内容
% 重新定义 \chapternumberline,使其不显示章节号
\renewcommand{\chapternumberline}[1]{}
\renewcommand{\chaptitlefont}{\normalfont\huge\bfseries}
% 重新定义 \printchaptertitle,使其只打印标题文本
\renewcommand{\printchaptertitle}[1]{\chaptitlefont #1}

\renewcommand{\printchaptertitle}[1]{%
  \chaptitlefont #1%
  % 设置 \leftmark 为章节标题文本,不带编号
  % \MakeUppercase{#1} 会将标题转换为大写,如果不需要,直接用 #1
  \markboth{\MakeUppercase{#1}}{}% 设置 \leftmark (左页眉)
}
\renewcommand{\sectionmark}[1]{%
  % 设置 \rightmark 为小节标题文本,不带编号
  % \MakeUppercase{#1} 会将标题转换为大写,如果不需要,直接用 #1
  \markright{\MakeUppercase{#1}}% 设置 \rightmark (右页眉)
}
\makeatother

\title{\huge\textbf{Analysis of Algorithms - Assignment 3}\vspace{-0.5cm}}
\author{\textbf{Zhenrui Zheng} \vspace{0.5cm} \\ \small Chinese University of Hong Kong, Shenzhen \\ \small\texttt{225040512@link.cuhk.edu.cn}}
\date{}
\setlength{\droptitle}{-1cm}

% 设置段落缩进为0
\setlength{\parindent}{0pt}
\setlength{\parskip}{1ex plus 0.5ex minus 0.2ex} % 可选:增加段落之间的垂直间距

% ===== 自定义命令 =====
\newtheorem{theorem}{Theorem}
\newtheorem{lemma}{Lemma}
\newtheorem{corollary}{Corollary}
\newtheorem{definition}{Definition}

\begin{document}

% ===== 标题页 & 目录 =====
\begin{titlingpage}
  \maketitle
  \renewcommand{\contentsname}{\huge Contents \vspace{-1cm}}
  \begin{KeepFromToc} % 将目录本身排除在目录之外
    \tableofcontents
  \end{KeepFromToc}
\end{titlingpage}

% ===== 章节模板 =====
\chapter{Amortized Analysis}

\section{Queue Implementation with Two Stacks}

\subsection{Implementation}

We implement a queue using two stacks by maintaining:
\begin{itemize}
    \item \textbf{Input Stack} ($S_{in}$): For enqueue operations
    \item \textbf{Output Stack} ($S_{out}$): For dequeue operations
\end{itemize}

\begin{algorithm}[H]
\caption{Queue Operations with Two Stacks}
\begin{algorithmic}[1]
\REQUIRE Two stacks $S_{in}$ and $S_{out}$
\STATE $\textbf{function Push}(x)$
    \STATE $S_{in}.\text{push}(x)$
\STATE \textbf{end function}

\STATE $\textbf{function Get()}$
    \IF{$S_{out}$ is empty}
        \WHILE{$S_{in}$ is not empty}
            \STATE $S_{out}.\text{push}(S_{in}.\text{pop}())$
        \ENDWHILE
    \ENDIF
    \RETURN $S_{out}.\text{pop}()$
\STATE \textbf{end function}
\end{algorithmic}
\end{algorithm}

\subsection{Proof of O(1) Amortized Time}
We use the \textit{accounting method} to prove the amortized cost analysis.

\textbf{Assigning Costs:} We assign an amortized cost of $4$ for each Push operation:
\begin{itemize}
    \item $1$ unit pays for the actual Push into $S_{in}$
    \item $3$ unit is stored as credit on the element for its future operations
\end{itemize}

For Get operations, we assign an amortized cost of $0$:
\begin{itemize}
    \item When $S_{out}$ is non-empty: the actual cost is $1$ (one Pop), and we use $1$ credit from the poped element (in $S_{out}$)
    \item When $S_{out}$ is empty: we transfer all elements from $S_{in}$ to $S_{out}$
    \begin{itemize}
        \item Each transfer requires one Pop from $S_{in}$ (cost $1$) and one Push to $S_{out}$ (cost $1$), paid by the credit on the element
        \item After transfer, each element in $S_{out}$ has $1$ credit left, which is enough to cover the cost of the future Pops.
              Then, we can simply Pop $1$ element from $S_{out}$ with cost $1$, covered by the credit on the element
        \item Total cost: $2k + 1$ for transferring $k$ elements and one final Pop, covered by $2k + 1$ credits
    \end{itemize}
\end{itemize}

\textbf{Invariant:} At any point, the total credit stored equals $3 \left| S_{in} \right|+\left| S_{out} \right|$, since each element in $S_{in}$ has $3$ units of credit from its Push operation,
and each element in $S_{out}$ has $1$ credit left after transferring from $S_{in}$, which cost $2$ credits.

\textbf{Conclusion:} Each Push has amortized cost $O(1)$, each Get has amortized cost $O(1)$, and the total amortized cost over any sequence of $n$ operations is $O(n)$, giving $O(1)$ amortized cost per operation.

\section{Amortized Computation Cost Analysis}

\textbf{Total Cost Calculation:}

Let $k = \lfloor \log_2(n) \rfloor$. The total cost is:
\begin{align*}
T(n) &= \sum_{j=0}^{k} 2^j + \left(n - (k + 1)\right) \cdot 1 \\
&= \sum_{j=0}^{k} 2^j + n - k - 1
\end{align*}

We know that $\sum_{j=0}^{k} 2^j = 2^{k+1} - 1 = 2 \cdot 2^k - 1$.

Since $2^k \leq n < 2^{k+1}$, we have:
\begin{align*}
2^k &\leq n \\
2^{k+1} &\leq 2n
\end{align*}

Therefore:
\begin{align*}
T(n) &= 2^{k+1} - 1 + n - k - 1 \\
&\leq 2n - 1 + n - k - 1 \\
&= 3n - k - 2 \\
&\leq 3n \\
\frac{T(n)}{n} &\leq \frac{3n}{n} = 3 = O(1)
\end{align*}

Thus, the amortized computation cost per day is $O(1)$.

\chapter{Element Selection Algorithm}

\section{Randomized Algorithm}

\subsection{Algorithm Description}

\begin{algorithm}[H]
\caption{Simplified Randomized Search}
\begin{algorithmic}[1]
\REQUIRE Arrays $X[1..n]$, $next[1..n]$, $c > 0$ and value $x$
\ENSURE True if $x \in X$, False otherwise
\STATE Let $m = c \lceil \sqrt{n} \rceil$
\STATE Randomly sample $m$ distinct indices uniformly from $[1..n]$
\STATE Among the sampled indices, find $i^*$ that minimizes $|X[i] - x|$
\STATE Follow $next$ pointers from $i^*$ for at most $m$ steps
    \FOR{$k = 1$ to $m$}
        \IF{$X[i^*] = x$}
            \RETURN True
        \ENDIF
        \STATE $i^* = next[i^*]$
    \ENDFOR
\RETURN False
\end{algorithmic}
\end{algorithm}

This requires $O(\sqrt{n})$ time complexity as expected.

\section{Success Probability Analysis}

Let $X'$ be the sorted version of $X$, and $x$ be the target element.
Let $I$ be the set of indices within distance $m=c\sqrt{n}$ from $x$ in $X'$, 
that is, $I = \{i \in [1..n] \mid \textit{lowerbound}(x)-i \leq m \pmod{n}\}$.
The probability that, if we take $m$ samples uniformly from $[1..n]$,
at least one of them is in $I$ is a hypergeometric distribution,
which is given by:
\begin{align*}
&P(\text{at least one sample among $m$ in $I$})  \\
&= 1 - P(\text{all $m$ samples in } \overline{I}) \\
&= 1 - \frac{\binom{m}{0} \binom{n-m}{m}}{\binom{n}{m}} \\
&= 1 - \frac{(n-m)!(n-m)!}{n!(n-2m)!} \quad \text{($m!$ cancels out)}\\
\end{align*}

We want this to be at least $0.99$:
\begin{align*}
1 - \frac{(n-m)!(n-m)!}{n!(n-2m)!} &\geq 0.99 \\
\frac{(n-m)!(n-m)!}{n!(n-2m)!} &\leq 0.01 \\
m &\leq 2.15 \sqrt{n}
\end{align*}

So $m = \lceil 2.15\sqrt{n} \rceil$ suffices for 99\% success probability.
Similarly, minimum $m$ for 99.9\% and 99.99\% success probabilities are $\lceil 2.63\sqrt{n} \rceil$ and $\lceil 3.04\sqrt{n} \rceil$, respectively.

\chapter{Shelf Scheduling Algorithm}

\section{NP-Hardness Proof}

We prove that the parallel scheduling problem is NP-hard by reduction from the Subset-sum problem under a special case, where $p=2$.

\textbf{Reduction:} Given an instance of subset-sum with a set of positive integers $\{a_1, a_2, \ldots, a_n\}$ and target sum $S$,
we construct a $2$-processor parallel scheduling instance as follows:
\begin{itemize}
    \item occupy time: $T[1,...,n]=\{a_1, a_2, ..., a_n\}$
    \item Processor: $P[1,...,n], P[i]=1$
\end{itemize}

The question becomes: Can we schedule these $n$ jobs on $2$ processors such that all jobs finish at time $t = S$?
This is equivalent to asking whether we can partition the $n$ elements into two subsets, with one of them sum up to $S$.

The reduction is given within polynomial time, and since Subset-sum is NP-complete, the parallel scheduling problem is NP-hard.

\section{3-Approximation Algorithm}

We present a scheduling algorithm that achieves a 3-approximation guarantee.
Since the performance bound of the algorithm given in the hint is difficult to prove,
I designed another very similar algorithm that can easily prove its 3-approximation property.

\begin{figure}[H]
    \centering
    \includegraphics[width=0.8\textwidth]{code/result/A3_P3.pdf}
    \caption{An example of the scheduling algorithm}
\end{figure}

\subsection{Algorithm Description:}
Since $p = 2^k$ for some integer $k$, there are only $\log_2 p$ distinct scales of processor requirements. We partition jobs into categories based on their processor requirements:
\begin{itemize}
    \item Category $i$: jobs with $P[j] \in (2^{i-1}, 2^i]$ for $i = 0, \ldots, k$
\end{itemize}

The algorithm proceeds as follows:
\begin{enumerate}
    \item categorize all jobs by processor requirements $P[j]$ into $k$ categories
    \item For each category $i$ from $k$ to $0$:
    \begin{itemize}
        \item Round up the processor requirements of all jobs in this category to $2^i$
        \item For each job in this category, find the processor with the earliest end time among all current processors,
        for example, processor $j$, and assign $p[j]\sim p[j+2^i-1]$ to this job.
        Such assignment would push the end time of these $2^i$ processors to the future.
    \end{itemize}
\end{enumerate}

\subsection{Time Complexity:}
Since there are in total $n$ jobs, and each requires searching the earliest end time among $p$ processors,
which can be done in $O(\log p)$ time, the total time complexity is $O(n \log p)$.

\section{Lower Bounds for Optimal Makespan}

Let $M^*$ be the optimal makespan. We establish two lower bounds:

\textbf{Bound 1:} $M^* \geq \max_{i=1}^{n} T[i]$ \\
This is trivial: any job $i$ must run for at least $T[i]$ time units. Therefore, the makespan cannot be smaller than the longest job duration.

\textbf{Bound 2:} $M^* \geq \frac{\sum_{i=1}^{n} T[i] \cdot P[i]}{p}$ \\
This inequality represents the relationship between total computation capacity and job requirements,
regardless of scheduling. The total computation capacity can be considered a large rectangle box,
which is $p \times M^*$, need to fit in all jobs, each can be considered a small rectangle box,
which is $P[i] \times T[i]$.

Thus, it is obvious that the volume of the large box is at least the sum of the volumes of all small boxes,
that is, $p \times M^* \geq \sum_{i=1}^{n} T[i] \cdot P[i]$, which gives $M^* \geq \frac{\sum_{i=1}^{n} T[i] \cdot P[i]}{p}$.

\section{3-Approximation Proof}

We decompose the schedule produced by our algorithm into two parts:
\[
\text{ALG} = A + B
\]
where:
\begin{itemize}
    \item $A$: total height of all non-last shelves across all categories
    \item $B$: total height of the last shelf in each category
\end{itemize}

\subsection{$B \leq \text{OPT}$}
Consider the last shelf in each category, which contains the largest duration job in that category,


\chapter{Randomized Algorithm}


% 标记最后一页用于总页数计算
\label{LastPage}

\end{document}
