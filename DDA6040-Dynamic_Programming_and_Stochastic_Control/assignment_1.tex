\documentclass[11pt, a4paper, oneside]{memoir}

% 数学公式包
\usepackage{amsmath, amssymb, amsthm}
\usepackage{mathtools}

% 算法包
\usepackage{algorithm}
\usepackage{algorithmic}

% 代码高亮
\usepackage{listings}
\usepackage{xcolor}

% 图表包
\usepackage{graphicx}
\usepackage{tikz}
\usepackage{pgfplots}
\pgfplotsset{compat=1.18}

% 表格包
\usepackage{booktabs}
\usepackage{array}
\usepackage{multirow}

% 中文支持
\usepackage[UTF8]{ctex}
\ctexset{proofname={Proof}}

% 其他实用包
\usepackage{hyperref}
\usepackage{geometry}
\usepackage{enumitem}
\usepackage{url}
\usepackage{float}
\usepackage{lipsum} % 随机文本包

% ===== 页面布局 =====
\setlrmarginsandblock{3cm}{3cm}{*} % 左边距3cm,右边距3cm
\setulmarginsandblock{2.5cm}{1.5cm}{*} % 上边距2.5cm,下边距1.5cm
\setlength{\beforechapskip}{0.5cm}
\checkandfixthelayout

% ===== 页面风格 =====
\makeevenfoot{headings}{}{\thepage}{}
\makeoddfoot{headings}{}{\thepage}{}
\makeevenhead{headings}{}{\leftmark}{}
\makeoddhead{headings}{}{\leftmark}{}
\makeheadrule{headings}{\textwidth}{0.4pt}

% ===== 章节格式 =====
\makeatletter
\renewcommand{\chaptername}{} % 移除"Chapter"字样
\renewcommand{\printchapternum}{} % 不打印章节号
\renewcommand{\afterchapternum}{} % 移除章节号后的内容
% 重新定义 \chapternumberline,使其不显示章节号
\renewcommand{\chapternumberline}[1]{}
\renewcommand{\chaptitlefont}{\normalfont\huge\bfseries}
% 重新定义 \printchaptertitle,使其只打印标题文本
\renewcommand{\printchaptertitle}[1]{\chaptitlefont #1}

\renewcommand{\printchaptertitle}[1]{%
  \chaptitlefont #1%
  % 设置 \leftmark 为章节标题文本,不带编号
  % \MakeUppercase{#1} 会将标题转换为大写,如果不需要,直接用 #1
  \markboth{\MakeUppercase{#1}}{}% 设置 \leftmark (左页眉)
}
\renewcommand{\sectionmark}[1]{%
  % 设置 \rightmark 为小节标题文本,不带编号
  % \MakeUppercase{#1} 会将标题转换为大写,如果不需要,直接用 #1
  \markright{\MakeUppercase{#1}}% 设置 \rightmark (右页眉)
}
\makeatother

\title{\huge\textbf{Dynamic Programming and Stochastic Control - Assignment 1}\vspace{-0.5cm}}
\author{\textbf{Zhenrui Zheng} \vspace{0.5cm} \\ \small Chinese University of Hong Kong, Shenzhen \\ \small\texttt{225040512@link.cuhk.edu.cn}}
\date{}
\setlength{\droptitle}{-1cm}

% 设置段落缩进为0
\setlength{\parindent}{0pt}
\setlength{\parskip}{1ex plus 0.5ex minus 0.2ex} % 可选:增加段落之间的垂直间距

% ===== 自定义命令 =====
\newtheorem{theorem}{Theorem}
\newtheorem{lemma}{Lemma}
\newtheorem{corollary}{Corollary}
\newtheorem{definition}{Definition}
\newcommand{\UPPER}{\text{UPPER}}
\newcommand{\OPEN}{\text{OPEN}}
\newcommand{\ParentOf}{\text{ParentOf}}

\begin{document}

% ===== 标题页 & 目录 =====
\begin{titlingpage}
  \maketitle
  \renewcommand{\contentsname}{\huge Contents \vspace{-1cm}}
  \begin{KeepFromToc} % 将目录本身排除在目录之外
    \tableofcontents
  \end{KeepFromToc}
\end{titlingpage}

% ===== 章节模板 =====
\chapter{Problem 1: Dynamic Pricing in Discrete Time}
\section{Derivation of Value Function}
Recall that in a Markov Decision Process with random state transition, the optimal action $a^*(s)$ in state $s$ is defined as:
\[ a^*(s) = \arg \max_{a \in A} Q(s,a) = \arg \max_{a \in A} \mathbb{E}_{s' \sim P(s'|s,a)} \left[ r(s',s,a) + V(s') \right] \]
For the given problem, the state of $k^\text{th}$ day is the remaining unsold items $x_k$,
and the action is the price $u_k$. Thus we can write the optimal action as:
\[ u_k^*\footnotemark = u^*(k, x_k) = \arg \max_{u \in \mathbb{R}_+} \mathbb{E}_{x_{k+1} \sim P(x_{k+1}|x_k,u)} \left[ g(x_{k+1},x_k,u_k) + V_{k+1}(x_{k+1}) \right] \]
Where $g(x_{k+1},x_k,u_k)$ is the profit from selling $x_k-x_{k+1}$ items at price $u$, that is:
\[ g(x_{k+1},x_k,u_k) = u_k(x_k-x_{k+1}) \]
and $V_{k+1} (x_{k+1})$ is the optimal expected value function of the next state $(k+1, x_{k+1})$.
Since we already have:
\[ q_k(u_k) = \alpha \exp (-u_k) \]
Thus the action value function $Q(x_k,u_k)$ for any $x_k > 0$ can be expanded (from its expectation form) as:
\begin{align*}
  Q(x_k,u_k) = & ~\alpha \exp (-u_k) \left[ g(x_k-1,x_k,u_k) + V_{k+1}(x_k-1) \right]  \\
               & + (1-\alpha \exp (-u_k)) \left[ g(x_k,x_k,u_k) + V_{k+1}(x_k) \right] \\
  =            & ~\alpha \exp (-u_k) \left[ u_k + V_{k+1}(x_k-1) \right]               \\
               & + (1-\alpha \exp (-u_k)) \left[ V_{k+1}(x_k) \right]
\end{align*}
To find the optimal $u_k$, we take the derivative of $Q(x_k,u_k)$ with respect to $u_k$ and set it to zero:
\begin{align*}
  \frac{\partial Q}{\partial u_k} = & -\alpha \exp (-u_k) \left[ u_k + V_{k+1}(x_k-1) \right]                                           \\
                                    & + \alpha \exp (-u_k) \cdot 1                                                                      \\
                                    & + \alpha \exp (-u_k) \left[ V_{k+1}(x_k) \right]                                                  \\
  =                                 & ~\alpha \exp (-u_k) \left[ 1 - u_k - V_{k+1}(x_k-1) + V_{k+1}(x_k) \right] = 0                    \\
  \implies                          & ~\alpha \exp (-u_k^*) \left[ u_k^* + V_{k+1}(x_k-1) - V_{k+1}(x_k) \right] = \alpha \exp (-u_k^*) \\
                                    & ~u^*_k = 1 + V_{k+1}(x_k) - V_{k+1}(x_k-1)
\end{align*}
Thus,
\begin{align*}
  V_k(x_k) = Q(x_k, u_k^*) & = \alpha \exp(-u_k^*) \left[ u_k^*(x_k) + V_{k+1}(x_k-1) - V_{k+1}(x_k) \right] + V_{k+1}(x_k) \\
                           & = \alpha \exp (-u_k^*) + V_{k+1}(x_k)
\end{align*}

\footnotetext{For simplicity, we use $u_k$ instead of $u_k(x_k)$ when there is no ambiguity.}

\section{Proof of Monotonicity}
Assume that $V_k(0) = 0$ and $V_{N}(x_{N}) = 0$, let's prove the closed form of $V_k(x_k)$ by induction.

Assume that for some $k < N$, $V_k(x_k)$ has the closed form:
\begin{align*}
  V_k(x_k) =
  \begin{cases}
    (N-k)\alpha \exp(-1)                                              & \text{if } x_k \geq N-k  \\
    \sum_{i=k}^{N-x_k} \alpha \exp(-u_i^*(x_k)) + x_k \alpha \exp(-1) & \text{if } 0 < x_k < N-k \\
    0                                                                 & \text{if } x_k = 0
  \end{cases}
\end{align*}

For $V_{k-1}$, there are four cases:
\begin{enumerate}
  \item $x_{k-1} = 0$ \\
        $V_{k-1}(x_{k-1}) = 0$ according to our assumption.
  \item $0 < x_{k-1} < N-k$
        \begin{align*}
          V_{k-1}(x_{k-1}) & = \alpha \exp(-u_{k-1}^*(x_{k-1})) + V_k(x_{k-1})                                                                  \\
                           & = \alpha \exp(-u_{k-1}^*(x_{k-1})) + \sum_{i=k}^{N-x_{k-1}} \alpha \exp(-u_i^*(x_{k-1})) + x_{k-1} \alpha \exp(-1) \\
                           & = \sum_{i=k-1}^{N-x_{k-1}} \alpha \exp(-u_i^*(x_{k-1})) + x_{k-1} \alpha \exp(-1)
        \end{align*}
  \item $x_{k-1} = N-k$ \\
        Let's first derive $u_{k-1}^*(x_{k-1})$. Assume that for some $k < N-1$, $u_k^*(N-k-1) = 1$, then:
        \begin{align*}
          u_{k-1}^*(N-k) & = 1 + V_k(N-k) - V_k(N-k-1)                                                                                   \\
                         & = 1 + (N-k)\alpha \exp(-1)                                                                                    \\
                         & \qquad - \left( \sum_{i=k}^{N-x_{k-1}+1} \alpha \exp(-u_i^*(x_{k-1}-1)) + (x_{k-1}-1) \alpha \exp(-1) \right) \\
                         & = 1 + (N-k)\alpha \exp(-1) - \left( \alpha \exp(-1) + (N-k-1) \alpha \exp(-1) \right)                         \\
                         & = 1
        \end{align*}
        Which also holds for $k=N-1$, $u_{N-1}^*(0) = 1$. Thus,
        \begin{align*}
          V_{k-1}(x_{k-1}) & = \alpha \exp(-u_{k-1}^*(x_{k-1})) + V_k(x_{k-1})     \\
                           & = \alpha \exp(-u_{k-1}^*(N-k)) + (N-k)\alpha \exp(-1) \\
                           & = \alpha \exp(-1) + (N-k) \alpha \exp(-1)             \\
                           & = (N-(k-1)) \alpha \exp(-1)
        \end{align*}
  \item $x_{k-1} \geq N-(k-1) \quad (x_{k-1} > N-k)$
        \begin{align*}
          u_{k-1}^*(x_{k-1}) & = 1 + V_k(x_{k-1}) - V_k(x_{k-1}-1)                                          \\
                             & = 1 + (N-k)\alpha \exp(-1) - (N-k)\alpha \exp(-1) \quad (x_{k-1}-1 \geq N-k) \\
                             & = 1
        \end{align*}
        Thus,
        \begin{align*}
          V_{k-1}(x_{k-1}) & = \alpha \exp(-u_{k-1}^*(x_{k-1})) + V_k(x_{k-1}) \\
                           & = \alpha \exp(-1) + (N-k)\alpha \exp(-1)          \\
                           & = (N-(k-1)) \alpha \exp(-1)
        \end{align*}
\end{enumerate}

Therefore, we have proved that the induction relation holds.
For $k=N$, It's trivial to prove that the assumption also holds by substitution. We have therefore proved the closed form of $V_k(x_k)$.

% Then, we can write the closed form of $u_k^*(x_k)$ as:
% \begin{align*}
%   u_k^*(x_k) &= 1 + V_{k+1}(x_k) - V_{k+1}(x_k-1) \\
%   &=
%   \begin{cases}
%     1 + (N-k)\alpha \exp(-1) - (N-k)\alpha \exp(-1) = 1 & \text{if } x_k-1 \geq N-k \\
%     1 + (N-k)\alpha \exp(-1) - \left( \alpha \exp(-1) + (N-k-1) \alpha \exp(-1) \right) = 1 & \text{if } x_k = N-k \\
%     1 + (x_k-(x_k-1))\alpha \exp(-1) - \alpha \exp(-u_{N-x_k}^*(x_k)) & \text{if } x_k < N-k
%   \end{cases}
% \end{align*}
% Let's expand the last case:
% \begin{align*}
%    ~& 1 + (x_k-(x_k-1))\alpha \exp(-1) - \alpha \exp(-u_{N-x_k}^*(x_k-1)) \\
%   =~& 1 + \alpha \left( \exp(-1) - \exp(-u_{N-x_k}^*(x_k-1)) \right) \\
%   =~& 1 + \alpha \left[ \exp(-1) - \exp \left[ - (1 + V_{N-x_k+1}(x_k-1) - V_{N-x_k+1}(x_k-2)) \right] \right] \\
% \end{align*}

Then, we can prove the monotonically non-increasing property of $u_k^*(x_k)$.
\begin{align*}
  u_k^*(x_k)              & = 1 + V_{k+1}(x_k) - V_{k+1}(x_k-1)               \\
  u_k^*(x_k)-u_k^*(x_k-1) & = V_{k+1}(x_k) - 2V_{k+1}(x_k-1) + V_{k+1}(x_k-2)
\end{align*}
Which actually requires us to prove that $V_{k+1}(x_k) - 2V_{k+1}(x_k-1) + V_{k+1}(x_k-2) \leq 0$,
that is, $V_k$ is a concave function. Since we have already derived the piecewise form of $V_k$,
we can prove its concavity by case analysis on $x_k$, expanding the expressions of $V_{k+1}$,
and using a little bit induction assumption.
However, this would be quite tedious and the proof is straightforward to those who understand the problem structure.

\chapter{Problem 2: Label correcting with negative arc lengths}
\section{Proof of Correctness}
Let's prove that the modified algorithm terminates with a shortest path from $s$ to $t$.
The proof is divided into two parts: firstly, we prove that the algorithm must terminate;
secondly, we prove that upon termination, the computed distance $d_t$ is indeed the shortest path distance.

\subsection*{Part A: Termination}
Let $d_j$ be the label of node $j$, representing the length of a computed path from $s$ to $j$.
The algorithm initializes $d_s = 0$ and $d_j = \infty$ for all $j \neq s$.

\begin{lemma}
  For any node $j$, the label $d_j$ is non-increasing throughout the execution of the algorithm.
\end{lemma}

\begin{proof}
  A label $d_j$ is updated only in the Modified Step 2, where it is set to $d_i + a_{ij}$.
  The condition for this update is $d_i + a_{ij} < \min\{d_j, \UPPER - u_j\}$, which implies $d_i + a_{ij} < d_j$.
  Thus, any update to $d_j$ strictly decreases its value.
\end{proof}

\begin{lemma}
  The label $d_j$ of any node $j$ is always greater than or equal to the shortest path distance from $s$ to $j$,
  denoted $\delta(s, j)$.
\end{lemma}

\begin{proof}
  We prove this by induction on the number of label updates.

  \textbf{Base Case:} Initially, $d_s = 0 = \delta(s, s)$ and $d_j = \infty \ge \delta(s, j)$ for $j \neq s$, the property holds.

  \textbf{Induction:} Assume that before any update, $d_k \ge \delta(s, k)$ for all nodes $k$.
  Consider an update to $d_j$ where $d_j$ is set to $d_i + a_{ij}$. By the inductive hypothesis,
  $d_i \ge \delta(s, i)$. By the triangle inequality for shortest paths, $\delta(s, j) \le \delta(s, i) + a_{ij}$.
  Thus, the new value for $d_j$ satisfies:
  \[ d_j^{\text{new}} = d_i + a_{ij} \ge \delta(s, i) + a_{ij} \ge \delta(s, j) \]
  The property is maintained after the update.
\end{proof}

\begin{theorem}
  The algorithm terminates.
\end{theorem}

\begin{proof}
  The algorithm terminates when the set $\OPEN$ is empty.
  A node $j$ is added to $\OPEN$ only when its label $d_j$ decreases.
  Therefore, we need to prove that each $d_j$ can only be decreased a finite number of times.

  From the previous lemma, we know that $d_j \geq \delta(s, j)$, and $d_j$ can only decrease.
  If we assume that arc lengths $a_{ij}$ are integers, then the number of possible values for $d_j$
  between its initial value and its lower bound $\delta(s, j)$ is finite.
  Therefore, the label of each node $j$ can only be updated a finite number of times,
  and thus can only be added to the $\OPEN$ set a finite number of times.
  Since the number of nodes is also finite, the algorithm must terminate.
\end{proof}

\subsection*{Part B: Correctness}
We now prove that upon termination, the algorithm finds a shortest path.
That is, the final label $d_t$ equals the shortest path distance $\delta(s, t)$.

\begin{theorem}
  If the graph $G$ contains no negative cycles and there is a path from $s$ to $t$,
  the modified label correcting algorithm terminates with $d_t = \delta(s, t)$.
\end{theorem}
\begin{proof}
  We know from Lemma 2 that $d_j \ge \delta(s, j)$ for all nodes $j$ throughout the algorithm.
  Thus, at termination, $d_t \ge \delta(s, t)$. We only need to prove that $d_t \le \delta(s, t)$.
  We prove this by contradiction. Assume the algorithm terminates with $d_t > \delta(s, t)$.
  Let $P^* = (s=v_0, v_1, \dots, v_k=t)$ be a shortest path from $s$ to $t$.
  Since $d_{v_0} = d_s = 0 = \delta(s, s)$ and we assume $d_{v_k} = d_t > \delta(s, t)$,
  there must be a first node $v_m$ on this path for which the final label $d_{v_m}$ satisfies $d_{v_m} > \delta(s, v_m)$.
  Since $v_m$ is the first such node on $P^*$, the label for the preceding node, $v_{m-1}$,
  must be correct at termination. That is, $d_{v_{m-1}} = \delta(s, v_{m-1})$.
  Because the algorithm has terminated, the set $\OPEN$ is empty.
  Which implies that node $v_{m-1}$ must have been removed from $\OPEN$ at some point after its label was last updated.
  Let's consider the last time $v_{m-1}$ was removed from $\OPEN$.
  At that time, its label was its final, optimal value, $d_{v_{m-1}} = \delta(s, v_{m-1})$.
  During the processing of $v_{m-1}$, the algorithm examined the arc $(v_{m-1}, v_m)$.
  Since the final label $d_{v_m}$ is not $\delta(s, v_m)$, the relaxation step for $v_m$ via $v_{m-1}$ must have failed.
  This means the condition for the update was not met:
  \[ d_{v_{m-1}} + a_{v_{m-1}, v_m} \ge \min\{d_{v_m}, \UPPER - u_{v_m}\} \]
  (where $d_{v_m}$ and $\UPPER$ are the values at that time).
  Since labels only decrease, the final values also satisfy this inequality.
  Substituting $d_{v_{m-1}} = \delta(s, v_{m-1})$ and noting that $\delta(s, v_{m-1}) + a_{v_{m-1}, v_m} = \delta(s, v_m)$
  because $(v_{m-1}, v_m)$ is an arc on a shortest path, we get:
  \[ \delta(s, v_m) \ge \min\{d_{v_m}, \UPPER - u_{v_m}\} \]
  This implies two conditions must hold:
  \begin{enumerate}
    \item $\delta(s, v_m) \ge d_{v_m}$
    \item $\delta(s, v_m) \ge \UPPER - u_{v_m}$
  \end{enumerate}
  Let's analyze the first condition. We already know from Lemma 2 that $d_{v_m} \ge \delta(s, v_m)$.
  Combining this with $\delta(s, v_m) \ge d_{v_m}$ implies that $d_{v_m} = \delta(s, v_m)$.
  This contradicts our assumption that $v_m$ was the first node on the path with $d_{v_m} > \delta(s, v_m)$.
  The pruning condition $\UPPER - u_j$ does not discard the shortest path: $\UPPER$ is the length of some $s-t$ path, so $\UPPER \ge \delta(s, t)$. The value $u_{v_m}$ is an underestimate of the distance from $v_m$ to $t$, so $u_{v_m} \le \delta(v_m, t)$.
  Therefore,
  \[ \UPPER - u_{v_m} \ge \delta(s, t) - \delta(v_m, t) \]
  Since $v_m$ is on a shortest path to $t$, we have $\delta(s, t) = \delta(s, v_m) + \delta(v_m, t)$. Substituting this gives:
  \[ \UPPER - u_{v_m} \ge (\delta(s, v_m) + \delta(v_m, t)) - \delta(v_m, t) = \delta(s, v_m) \]
  This shows that $\delta(s, v_m) \le \UPPER - u_{v_m}$ is always true.
  The pruning step only prevents an update if $\delta(s, v_m) \ge \UPPER - u_{v_m}$, which can only happen if equality holds.
  This pruning is safe because if $d_i + a_{ij} + u_j \ge \UPPER$,
  the path through $(i,j)$ cannot lead to a better solution than the one that defined the current $\UPPER$.

  To this end, we have proved that the assumption that a node on the shortest path has a suboptimal label at termination is false.
  Thus, all nodes on the shortest path, including $t$, must have their optimal labels.
  That is, upon termination, $d_t = \delta(s, t)$.
\end{proof}

\newpage
\section{Label Correcting Algorithm as a Special Case}
We can see that the Label Correcting Algorithm given in class is a special case of the modified algorithm
through comparing their update rules:
\begin{itemize}
  \item \textbf{Standard LCA Update Rule:}
        \[ d_i + a_{ij} < \min\{d_j, \UPPER\} \]
  \item \textbf{Modified LCA Update Rule:}
        \[ d_i + a_{ij} < \min\{d_j, \UPPER - u_j\} \]
\end{itemize}
it's straightforward that setting $u_j$ to zero in modified LCA will lead to standard LCA.
But it's also worth noting that setting $u_j=0$ implies that $\delta(j,t)>=0$ for any node $j$,
which is always true only when arc length $\geq 0$, which is consistent with the conditions for standard LCA.

\chapter{Problem 3: Object Partition as Shortest Path Problem}
We can model this problem as finding the shortest path in a directed acyclic graph (DAG).
Let's construct the graph $G = (V, E)$ as follows:
\begin{itemize}
  \item \textbf{Nodes:} We define a set of $N+1$ nodes, $V = \{v_0, v_1, \dots, v_N\}$.
        Each node $v_k$ represents a potential boundary point in our partition.
        Specifically, node $v_k$ represents making a partition after the $k$-th object.
        $v_0$ represents before the 1st object.
  \item \textbf{Edges:} For every pair $(i, j)$ such that $0 \le i < j \le N$,
        we draw a directed edge from node $v_i$ to node $v_j$.
        This edge $(v_i, v_j)$ represents forming a single cluster containing objects $\{i+1, i+2, \dots, j\}$.
  \item \textbf{Edge Weights:} The weight of edge $(v_i, v_j)$, denoted as $w(v_i, v_j)= a_{i+1, j}$,
        is the cost of generating the corresponding cluster.
  \item \textbf{Node Distance:} The distance of node $v_i$, denoted as $d_{v_i}$,
        can be seen as the minimum cost to generate a valid partition for objects $1~i$.
        Thus, the distance of node $v_N$ is the minimum cost for partitioning all objects.
\end{itemize}
Therefore, the problem of finding the minimum cost grouping is equivalent to finding the \textbf{shortest path}
from node $v_0$ to node $v_N$ in the constructed graph $G$. We can either solve it using any shortest path algorithm,
or in a DP style. For the latter, the state transition is:
\begin{align*}
  d_{v_i} = \min_{0 \le j < i} \{d_{v_j} + a_{j+1, i}\}
\end{align*}
and $d_{v_0} = 0$. Iteration over all nodes $v_1~v_N$ will give the minimum cost for partitioning all objects.
To recover the actual grouping, one can record the predecessor for each node in shortest-path solution,
or record the $\arg \min_{j} \{d_{v_j} + a_{j+1, i}\}$ for each $i$ during Iteration.

\begin{algorithm}[H]
  \caption{Optimal Clustering DP}
  \label{alg:clustering}
  \begin{algorithmic}[1]
    \STATE \textbf{Input:} Number of objects $N$, and costs $a_{ij}$ for $1 \le i \le j \le N$.
    \STATE \textbf{Output:} The minimum total cost and the corresponding partition.
    \STATE Initialize arrays $C[0 \dots N]$ and $P[1 \dots N]$.
    \STATE $C[0] \leftarrow 0$
    \STATE \textbf{for} $j \leftarrow 1$ \textbf{to} $N$ \textbf{do}
    \STATE \quad $C[j] \leftarrow \infty$
    \STATE \quad \textbf{for} $i \leftarrow 0$ \textbf{to} $j-1$ \textbf{do}
    \STATE \quad \quad \textbf{if} $C[i] + a_{i+1, j} < C[j]$ \textbf{then}
    \STATE \quad \quad \quad $C[j] \leftarrow C[i] + a_{i+1, j}$
    \STATE \quad \quad \quad $P[j] \leftarrow i$
    \STATE \quad \quad \textbf{end if}
    \STATE \quad \textbf{end for}
    \STATE \textbf{end for}
    \STATE \COMMENT{The minimum cost is $C[N]$. Now, reconstruct the clusters.}
    \STATE Initialize an empty list of clusters.
    \STATE $j \leftarrow N$
    \STATE \textbf{while} $j > 0$ \textbf{do}
    \STATE \quad $i \leftarrow P[j]$
    \STATE \quad Add cluster $\{i+1, \dots, j\}$ to the clusters.
    \STATE \quad $j \leftarrow i$
    \STATE \textbf{end while}
    \STATE \textbf{return} $C[N]$ and clusters.
  \end{algorithmic}
\end{algorithm}

\chapter{Problem 4: Dynamic Programming with Discount Factor}
To derive the given DP algorithm, we will use the Principle of Optimality.
We define the value function $V_k(x_k)$ as the minimum expected discounted cost-to-go from stage $k$ to stage $N$,
given that the system is in state $x_k$ at time $k$.

According to the definition, $V_k(x_k)$ should be:
\[ V_k(x_k) = \min_{\mu_k, \dots, \mu_{N-1}} E_{w_k, \dots, w_{N-1}} \left( \sum_{j=k}^{N-1} \alpha^{j-k} g_j(x_j, \mu_j(x_j), w_j) + \alpha^{N-k} g_N(x_N) \right) \]
This definition implies that $V_k(x_k)$ represents the minimum expected future cost,
where each instantaneous cost $g_j$ is discounted by $\alpha^{j-k}$ relative to stage $k$.

For the terminal stage $k=N$, the summation $\sum_{j=N}^{N-1}$ is empty. According to our definition of $V_N(x_N)$:
\[ V_N(x_N) = \min E \left( \alpha^{N-N} g_N(x_N) \right) = g_N(x_N) \]
Which matches the first equation provided in the problem statement.

For any stage $k < N$, we apply the Principle of Optimality. The optimal cost $V_k(x_k)$ can be expressed by considering the immediate cost at stage $k$ and the optimal future cost from stage $k+1$.
\begin{align*}
  V_k(x_k) = \min_{u_k \in U_k(x_k)} E_{w_k} \left[ \alpha^{k-k} g_k + \min_{\mu_{k+1}, \dots, \mu_{N-1}} E_{w_{k+1}, \dots, w_{N-1}} \left( \sum_{j=k+1}^{N-1} \alpha^{j-k} g_j + \alpha^{N-k} g_N \right) \right]
\end{align*}
Where we omit the dependence of $g_k(x_k, u_k, w_k)$ for simplicity. In which, the second term in the expectation:
\[ \min_{\mu_{k+1}, \dots, \mu_{N-1}} E_{w_{k+1}, \dots, w_{N-1}} \left( \sum_{j=k+1}^{N-1} \alpha^{j-k} g_j + \alpha^{N-k} g_N \right) \]
We can factor out $\alpha$ from each term in the sum and the terminal cost:
\begin{align*}
   & = \min_{\mu_{k+1}, \dots, \mu_{N-1}} E_{w_{k+1}, \dots, w_{N-1}} \left( \alpha \sum_{j=k+1}^{N-1} \alpha^{j-(k+1)} g_j + \alpha \alpha^{N-(k+1)} g_N \right) \\
   & = \alpha \min_{\mu_{k+1}, \dots, \mu_{N-1}} E_{w_{k+1}, \dots, w_{N-1}} \left( \sum_{j=k+1}^{N-1} \alpha^{j-(k+1)} g_j + \alpha^{N-(k+1)} g_N \right)
\end{align*}
Which matches exactly with previously defined $V_{k+1}(x_{k+1})$.
Thus, we can re-write the expression as:
\begin{align*}
  V_k(x_k) & = \min_{u_k \in U_k(x_k)} E_{w_k} \left[ \alpha^{k-k} g_k + \alpha V_{k+1}(x_{k+1}) \right]              \\
           & = \min_{u_k \in U_k(x_k)} E_{w_k} \left[ g_k(x_k, u_k, w_k) + \alpha V_{k+1}(f_k(x_k, u_k, w_k)) \right]
\end{align*}
Which matches exactly with the given DP algorithm.

\chapter{Problem 5: Dynamic Programming with Conditional Termination}
To reformulate the problem into the basic problem framework,
we need to define a new augmented state, augmented system dynamics,
and augmented cost functions that capture the termination logic and costs.

Let the original state be $x_k$. We augment the state $x_k$ with a binary flag $s_k$ indicating whether the system has terminated.
Thus the new state is $\tilde{x}_k = (x_k, s_k)$.
When $s_k = \text{TRUE}$, the specific value of $x_k$ is no longer relevant for future evolution or costs.
We can introduce a dummy state $x_{\text{dummy}}$ for this component.

Then, the new system dynamics can be written as:
\begin{align*}
  \tilde{x}_{k+1} = \begin{cases}
                      (x_{\text{dummy}}, \text{TRUE})    & \text{if } s_k = \text{TRUE} \text{ or } w_k = \bar{w} \text{ or } u_k = \bar{u} \\
                      (f_k(x_k, u_k, w_k), \text{FALSE}) & \text{otherwise}
                    \end{cases}
\end{align*}

The new control policy is:
\begin{align*}
  \tilde{u}_k(\tilde{x}_k) = \begin{cases}
                               u_{\text{dummy}} & \text{if } s_k = \text{TRUE} \\
                               u_k(x_k)         & \text{otherwise}
                             \end{cases}
\end{align*}
Where $u_{\text{dummy}}$ acts as a padding strategy, only used to ensure the consistency of the formula.

The new cost function is:
\begin{align*}
  \tilde{g}_k(\tilde{x}_k, \tilde{u}_k, w_k) = \begin{cases}
                                                 0                      & \text{if } s_k = \text{TRUE}                                                         \\
                                                 g_k(x_k, u_k, w_k) + T & \text{if } s_k = \text{FALSE} \text{ and } (w_k = \bar{w} \text{ or } u_k = \bar{u}) \\
                                                 g_k(x_k, u_k, w_k)     & \text{otherwise}
                                               \end{cases}
\end{align*}

To this end, we have reformulated the problem into the basic problem framework.
One can directly use the formula from the basic problem to solve this problem with a termination state:
\begin{align*}
  E\left[\tilde{g}_N(\tilde{x}_N) + \sum_{k=i}^{N-1} \tilde{g}_k(\tilde{x}_k, \tilde{u}_k(\tilde{x}_k), w_k) \right]
\end{align*}

\chapter{Problem 6: Unbounded Knapsack Problem}
To handle the product maximization,
we transform it into a sum maximization by taking the natural logarithm of the objective function.
Since $p_j(m_j) \in (0, 1]$ (assuming non-zero success probabilities), $\log(p_j(m_j)) \leq 0$.
Maximizing the sum of these logarithms is equivalent to maximizing their product.
Let $g_j(m_j) = \log(p_j(m_j))$. The transformed problem is:

\[ \text{Maximize } \sum_{j=1}^N g_j(m_j) \]
\[ \text{Subject to } \sum_{j=1}^N c_j m_j \leq A, \quad m_j \in \{0, 1, 2, \ldots\} \]

Which is a classic unbounded knapsack problem.

Let $dp(j, k)$ represent the maximum total log-reliability for the first $j$ stages (i.e., stages $1, 2, \ldots, j$), given that a total cost of exactly $k$ has been spent on spare components for these $j$ stages.
\begin{itemize}
  \item $j$: The current stage index, ranging from $0$ to $N$.
  \item $k$: The total cost spent on spare components for stages $1$ through $j$, ranging from $0$ to $A$.
\end{itemize}
Then we have base cases:
\begin{itemize}
  \item $dp(0, 0) = 0$: represents the state before considering any stages,
        where no cost has been spent and the accumulated log-reliability is $0$ (an empty product is $1$).
  \item $dp(0, k) = -\infty$ for $k = 1, \ldots, A$: Any state with a non-zero cost before processing any stages is unreachable.
        We use $-\infty$ to denote an unreachable state or a state with zero reliability (since $\log(0) = -\infty$).
\end{itemize}
The state transition would be:
\[ dp(j, k) = \max_{0 \leq m_j \leq \lfloor k / c_j \rfloor} \{ dp(j-1, k - c_j m_j) + g_j(m_j) \} \]
The maximization is performed over all valid integer values of $m_j$ such that $c_j m_j \leq k$
(to ensure $k - c_j m_j \geq 0$) and $m_j \geq 0$. We initialize $dp(j, k) = -\infty$ for all $j, k$ before computation.
We only consider terms where $dp(j-1, k - c_j m_j) \neq -\infty$.

After filling the $DP$ table up to stage $N$, the maximum total log-reliability for the entire device,
considering all possible budget allocations up to $A$, is:
\[ \max_{0 \leq k \leq A} dp(N, k) \]
The maximum overall device reliability is then obtained by exponentiating this value:
\[ \text{Maximum Reliability} = e^{\max_{0 \leq k \leq A} dp(N, k)} \]
To reconstruct the optimal solution, one can store the optimal $m_j$ for each $j, k$ during iteration,
and then backtrack to find the optimal solution.

\chapter{Problem 7: Monotonicity Property of Dynamic Programming}
We need to prove two statements:
\begin{enumerate}
  \item If $J_{N-1}(x) \le J_N(x)$ for all $x \in S$, then $J_k(x) \le J_{k+1}(x)$ for all $x \in S$ and $k \in \{0, \dots, N-1\}$.
  \item If $J_{N-1}(x) \ge J_N(x)$ for all $x \in S$, then $J_k(x) \ge J_{k+1}(x)$ for all $x \in S$ and $k \in \{0, \dots, N-1\}$.
\end{enumerate}
The core of the proof relies on the monotonicity of the Bellman operator:

For a time-invariant system, we can define the Bellman operator $T$ as:
$T(J)(x) = \min_{u \in U(x)} E_w[g(x, u, w) + J(f(x, u, w))]$
The DP recursion can then be written as $J_k(x) = T(J_{k+1})(x)$.

\begin{lemma}
  If two functions $J(y)$ and $\bar{J}(y)$ satisfy $J(y) \le \bar{J}(y)$ for all $y \in S$, then $T(J)(x) \le T(\bar{J})(x)$ for all $x \in S$.
\end{lemma}

\begin{proof}
  Assume $J(y) \le \bar{J}(y)$ for all $y \in S$.
  For any given state $x \in S$ and control $u \in U(x)$, we have:
  $g(x, u, w) + J(f(x, u, w)) \le g(x, u, w) + \bar{J}(f(x, u, w))$.
  Taking the expectation with respect to $w$ (which preserves inequalities):
  $E_w[g(x, u, w) + J(f(x, u, w))] \le E_w[g(x, u, w) + \bar{J}(f(x, u, w))]$.
  Let $Q_J(x, u) = E_w[g(x, u, w) + J(f(x, u, w))]$ and $Q_{\bar{J}}(x, u) = E_w[g(x, u, w) + \bar{J}(f(x, u, w))]$.
  Thus, $Q_J(x, u) \le Q_{\bar{J}}(x, u)$ for all $u \in U(x)$.

  Now, consider the definition of the Bellman operator:
  $T(J)(x) = \min_{u \in U(x)} Q_J(x, u),T(\bar{J})(x) = \min_{u \in U(x)} Q_{\bar{J}}(x, u)$.
  Let $u^*$ be an optimal control for $T(\bar{J})(x)$, so $T(\bar{J})(x) = Q_{\bar{J}}(x, u^*)$.
  Since $Q_J(x, u) \le Q_{\bar{J}}(x, u)$ for all $u \in U(x)$, it follows that $Q_J(x, u^*) \le Q_{\bar{J}}(x, u^*)$.
  By the definition of the minimum, $T(J)(x) \le Q_J(x, u^*)$.
  Combining these inequalities, we get:
  $T(J)(x) \le Q_J(x, u^*) \le Q_{\bar{J}}(x, u^*) = T(\bar{J})(x)$.
  Therefore, the monotonicity of Bellman operator $T$ is guaranteed.
\end{proof}

\begin{lemma}
  If $J_{N-1}(x) \le J_N(x)$ for all $x \in S$, then $J_k(x) \le J_{k+1}(x)$ for all $x \in S$ and $k \in \{0, \dots, N-1\}$.
\end{lemma}

\begin{proof}
  We will prove this by backward induction on $k$.

  \textbf{Inductive Hypothesis}:
  Assume that for some $k \in \{0, \dots, N-2\}$, we have $J_{k+1}(x) \le J_{k+2}(x)$ for all $x \in S$.

  \textbf{Inductive Step}:
  We need to show that $J_k(x) \le J_{k+1}(x)$ for all $x \in S$.
  From the Bellman equation, we have:
  \begin{align*}
    J_k(x)     & = T(J_{k+1})(x) \\
    J_{k+1}(x) & = T(J_{k+2})(x)
  \end{align*}
  By the inductive hypothesis, $J_{k+1}(y) \le J_{k+2}(y)$ for all $y \in S$.
  Since the Bellman operator $T$ is monotonic, applying $T$ to both sides of the inequality $J_{k+1} \le J_{k+2}$ preserves the inequality:
  $T(J_{k+1})(x) \le T(J_{k+2})(x)$.
  Substituting the definitions of $J_k(x)$ and $J_{k+1}(x)$ from the Bellman equation:
  $J_k(x) \le J_{k+1}(x)$.
  This completes the induction.
\end{proof}

\begin{lemma}
  If $J_{N-1}(x) \ge J_N(x)$ for all $x \in S$, then $J_k(x) \ge J_{k+1}(x)$ for all $x \in S$ and $k \in \{0, \dots, N-1\}$.
\end{lemma}

\begin{proof}
  This can be proven using the exact same steps as above.

  \textbf{Inductive Hypothesis}:
  Assume that for some $k \in \{0, \dots, N-2\}$, we have $J_{k+1}(x) \ge J_{k+2}(x)$ for all $x \in S$.

  \textbf{Inductive Step}:
  We need to show that $J_k(x) \ge J_{k+1}(x)$ for all $x \in S$.
  From the Bellman equation, we have:
  \begin{align*}
    J_k(x)     & = T(J_{k+1})(x) \\
    J_{k+1}(x) & = T(J_{k+2})(x)
  \end{align*}
  By the inductive hypothesis, $J_{k+1}(y) \ge J_{k+2}(y)$ for all $y \in S$.
  Since the Bellman operator $T$ is monotonic, applying $T$ to both sides of the inequality $J_{k+1} \ge J_{k+2}$ preserves the inequality:
  $T(J_{k+1})(x) \ge T(J_{k+2})(x)$.
  Substituting the definitions of $J_k(x)$ and $J_{k+1}(x)$ from the Bellman equation:
  $J_k(x) \ge J_{k+1}(x)$.
  This completes the induction.
\end{proof}

To this end, we have proved the monotonicity property of DP.

\chapter{Problem 8: Multiplicative Cost Dynamic Programming}
In the framework of a basic finite-horizon optimal control problem, we consider a system with state dynamics given by:
\[
  x_{k+1} = f_k(x_k, u_k, w_k), \quad k = 0, 1, \dots, N-1
\]
where $x_k$ is the state at stage $k$, $u_k$ is the control chosen at stage $k$,
and $w_k$ is a random disturbance occurring at stage $k$. The initial state $x_0$ is given.

The problem specifies a multiplicative cost function:
\[
  E_w[g_N(x_N)\prod_{k=1}^{N}g_k(x_k,u_k,w_k)]
\]
To provide a well-posed DP-like algorithm within the ``basic problem framework'', we interpret the cost function as follows:
\begin{itemize}
  \item $g_N(x_N)$ is the terminal cost, dependent only on the final state $x_N$.
  \item The product term $\prod_{k=0}^{N-1} g_k(x_k, u_k, w_k)$ represents the product of stage costs, where $g_k(x_k, u_k, w_k)$ is the cost factor incurred at stage $k$ due to state $x_k$, control $u_k$, and disturbance $w_k$. The original problem's indexing ($k=1$ to $N$ in the product) is adjusted to $k=0$ to $N-1$ to align with standard finite-horizon DP where controls $u_0, \dots, u_{N-1}$ are chosen.
\end{itemize}
Thus, the total cost function to be minimized is interpreted as:
\[
  J = E_{w_0, \dots, w_{N-1}}\left[g_N(x_N) \cdot \prod_{k=0}^{N-1} g_k(x_k, u_k, w_k)\right]
\]
The objective is to find a policy $\pi = \{\mu_0, \dots, \mu_{N-1}\}$ such that $u_k = \mu_k(x_k)$ minimizes $J$.
The problem states that $g_k(x_k, u_k, w_k) \ge 0$ for all $x_k, u_k, w_k$,
which is crucial for the well-definedness of the minimization problem with a multiplicative cost.

\section{Dynamic Programming Algorithm}
We develop a dynamic programming algorithm using a backward pass approach. Let $J_k(x_k)$ denote the minimum expected cost from stage $k$ to the terminal stage $N$, given that the system is in state $x_k$ at stage $k$.
\subsection*{1. Terminal Stage ($N$)}
At the terminal stage $N$, no further controls are chosen. The remaining cost is simply the terminal cost $g_N(x_N)$.
\[
  J_N(x_N) = g_N(x_N)
\]
for all possible states $x_N$.
\subsection*{2. Iteration for Stages $k = N-1, N-2, \dots, 0$ (Backward Pass)}
For each stage $k$ (from $N-1$ down to $0$) and for each possible state $x_k$:
The decision is to choose a control $u_k$. This choice incurs an immediate multiplicative cost factor $g_k(x_k, u_k, w_k)$ and transitions the system to a new state $x_{k+1} = f_k(x_k, u_k, w_k)$. The minimum expected future cost from $x_{k+1}$ is $J_{k+1}(x_{k+1})$.
Since the cost is multiplicative and all $g_k \ge 0$, the total expected cost from stage $k$ onwards, given $x_k$ and $u_k$, is the expectation of the product of the current cost factor and the future minimum expected cost.
The Bellman equation for this problem is:
\[
  J_k(x_k) = \min_{u_k \in U_k(x_k)} E_{w_k}[g_k(x_k, u_k, w_k) \cdot J_{k+1}(f_k(x_k, u_k, w_k))]
\]
where $U_k(x_k)$ is the set of admissible controls at state $x_k$ at stage $k$.
The optimal control policy $\mu_k(x_k)$ for state $x_k$ at stage $k$ is stored as:
\[
  \mu_k(x_k) = \arg\min_{u_k \in U_k(x_k)} E_{w_k}[g_k(x_k, u_k, w_k) \cdot J_{k+1}(f_k(x_k, u_k, w_k))]
\]
% If the disturbance $w_k$ is a discrete random variable with a known probability distribution $P(w_k)$,
% the expectation can be computed as a sum:
% \[
% E_{w_k}[\dots] = \sum_{w_k} P(w_k) \cdot [g_k(x_k, u_k, w_k) \cdot J_{k+1}(f_k(x_k, u_k, w_k))]
% \]
% If $w_k$ is a continuous random variable, the expectation would be an integral.

After computing $J_k(x_k)$ and $\mu_k(x_k)$ for all stages $k=N-1, \dots, 0$ and all relevant states $x_k$,
the minimum expected total cost from the initial state $x_0$ is $J_0(x_0)$.
The optimal policy is given by the sequence of functions $\pi^* = \{\mu_0, \mu_1, \dots, \mu_{N-1}\}$.

One might be tempted to use a seemingly viable transformation by taking the logarithm of the total cost $J$
and attempting to convert it directly into a standard additive DP problem:
\begin{align*}
  \log(J) & = \log \left( E_{w_0, \dots, w_{N-1}}\left[g_N(x_N) \cdot \prod_{k=0}^{N-1} g_k(x_k, u_k, w_k)\right] \right) \\
          & = E_{w_0, \dots, w_{N-1}}\left[\log(g_N(x_N)) + \sum_{k=0}^{N-1} \log(g_k(x_k, u_k, w_k))\right]
\end{align*}
however, this is incorrect because the logarithm is a non-linear function that does not commute with the expectation operator, meaning $\log(E[X]) \neq E[\log X]$.

\chapter{Problem 9: Flexible Employment}
\textit{The original solution title was ``Optimal Job Reception Problem'', but the term ``灵活就业'' suddenly came to my mind. Quite interesting anyway.}

Let $P(w_j)$ be the probability of receiving a job offer with salary $w_j$, $c$ be the unemployment benefit,
and $\alpha < 1$ be the discount factor. Define the following value functions:
\begin{itemize}
  \item $V(\bar{s}_j)$: The maximum expected discounted income if the worker is employed at salary $w_j$.
  \item $V(s_j)$: The maximum expected discounted income if the worker is unemployed and receives an offer $w_j$.
  \item $V_U$: The maximum expected discounted income if the worker is unemployed and has not yet received an offer (i.e., before the current period's offer is revealed).
\end{itemize}

\section{Part (a)}
If the worker accepts a job with salary $w_j$, she (why is ``she''?) will earn $w_j$ indefinitely.
The total discounted income is an infinite geometric series:
\[ V(\bar{s}_j) = w_j + \alpha w_j + \alpha^2 w_j + \dots = \sum_{k=0}^\infty \alpha^k w_j = \frac{w_j}{1-\alpha} \]

When being unemployed and offred $w_j$, The worker has two options:
\begin{enumerate}
  \item \textbf{Accept $w_j$:} She receives $w_j$ in the current period and transitions to the employed state $\bar{s}_j$
        in the next period. The total value is $w_j + \alpha V(\bar{s}_j)$. Substituting $V(\bar{s}_j)$:
        \[ w_j + \alpha \frac{w_j}{1-\alpha} = w_j \left(1 + \frac{\alpha}{1-\alpha}\right) = w_j \left(\frac{1-\alpha+\alpha}{1-\alpha}\right) = \frac{w_j}{1-\alpha} \]
  \item \textbf{Reject $w_j$:} She receives unemployment benefit $c$ in the current period and remains unemployed,
        waiting for a new offer in the next period. The total value is $c + \alpha V_U$.
\end{enumerate}
Thus, the Bellman equation for $V(s_j)$ is:
\[ V(s_j) = \max \left( \frac{w_j}{1-\alpha}, \quad c + \alpha V_U \right) \]
Where $V_U$ is the expected value of $V(s_k)$ over all possible offers $w_k$:
\[ V_U = \sum_{k=1}^n P(w_k) V(s_k) \]
Substituting $V(s_k)$:
\[ V_U = \sum_{k=1}^n P(w_k) \max \left( \frac{w_k}{1-\alpha}, \quad c + \alpha V_U \right) \]
This is an implicit equation for $V_U$. Due to the monotonicity of the $\max$ function and $\alpha < 1$,
this equation has a unique fixed-point solution for $V_U$.

The worker will accept the job offer $w_j$ if the value of accepting is greater than or equal to the value of rejecting:
\[ \frac{w_j}{1-\alpha} \ge c + \alpha V_U \]
Multiplying both sides by $(1-\alpha)$:
\[ w_j \ge (1-\alpha)(c + \alpha V_U) \]
Let $\bar{w} = (1-\alpha)(c + \alpha V_U)$.
The optimal strategy is to accept a job offer $w_j$ if and only if $w_j \ge \bar{w}$.
This proves the existence of such a threshold.

\section{Part (b)}
\subsection*{Case 1: Firing probability $p_i = p$ for all $i$}
If employed at $w_j$, the worker earns $w_j$ in the current period.
In the next period, with probability $1-p$, she remains employed at $w_j$,
and with probability $p$, she is fired and returns to the unemployed state (before receiving a new offer).
\[ V(\bar{s}_j) = w_j + \alpha \left[ (1-p)V(\bar{s}_j) + p V_U \right] \]
Solving for $V(\bar{s}_j)$:
\begin{align*} V(\bar{s}_j) - \alpha(1-p)V(\bar{s}_j) &= w_j + \alpha p V_U \\ V(\bar{s}_j) (1 - \alpha(1-p)) &= w_j + \alpha p V_U \\ V(\bar{s}_j) &= \frac{w_j + \alpha p V_U}{1 - \alpha(1-p)} \end{align*}

The Bellman equation for $V(s_j)$ remains:
\[ V(s_j) = \max \left( w_j + \alpha V(\bar{s}_j), \quad c + \alpha V_U \right) \]

The worker accepts $w_j$ if $w_j + \alpha V(\bar{s}_j) \ge c + \alpha V_U$.
Substitute $V(\bar{s}_j)$:
\begin{align*}
  w_j + \alpha \frac{w_j + \alpha p V_U}{1 - \alpha(1-p)}                       & \ge c + \alpha V_U                                                                                 \\
  w_j \left(1 + \frac{\alpha}{1 - \alpha(1-p)}\right)                           & \ge c + \alpha V_U - \frac{\alpha^2 p V_U}{1 - \alpha(1-p)}                                        \\
  w_j \left(\frac{1 - \alpha + \alpha p + \alpha}{1 - \alpha + \alpha p}\right) & \ge c + V_U \left(\frac{\alpha(1 - \alpha + \alpha p) - \alpha^2 p}{1 - \alpha + \alpha p}\right)  \\
  w_j \left(\frac{1 + \alpha p}{1 - \alpha + \alpha p}\right)                   & \ge c + V_U \left(\frac{\alpha - \alpha^2 + \alpha^2 p - \alpha^2 p}{1 - \alpha + \alpha p}\right) \\
  w_j \left(\frac{1 + \alpha p}{1 - \alpha + \alpha p}\right)                   & \ge c + V_U \left(\frac{\alpha(1 - \alpha)}{1 - \alpha + \alpha p}\right)                          \\
\end{align*}
Multiplying both sides by $(1 - \alpha + \alpha p)$:
\begin{align*}
  w_j (1 + \alpha p) & \ge c(1 - \alpha + \alpha p) + V_U \alpha(1 - \alpha)                      \\
  w_j                & \ge \frac{c(1 - \alpha + \alpha p) + V_U \alpha(1 - \alpha)}{1 + \alpha p}
\end{align*}
Let $\bar{w}' = \frac{c(1 - \alpha + \alpha p) + V_U \alpha(1 - \alpha)}{1 + \alpha p}$.
This $\bar{w}'$ is a constant threshold (once $V_U$ is determined).
Therefore, the result from part (a) still holds: there exists a single threshold $\bar{w}'$,
and the worker accepts $w_j$ if and only if $w_j \ge \bar{w}'$. $V_U$ is still determined by its fixed-point equation,
incorporating the new $V(\bar{s}_k)$.

\subsection*{Case 2: Firing probability $p_i$ depends on $i$}
The firing probability is now $p_j$, specific to salary $w_j$.
\[ V(\bar{s}_j) = w_j + \alpha \left[ (1-p_j)V(\bar{s}_j) + p_j V_U \right] \]
Solving for $V(\bar{s}_j)$:
\[ V(\bar{s}_j) = \frac{w_j + \alpha p_j V_U}{1 - \alpha(1-p_j)} \]
$V(s_j)$:
\[ V(s_j) = \max \left( w_j + \alpha V(\bar{s}_j), \quad c + \alpha V_U \right) \]
The worker accepts $w_j$ if $w_j + \alpha V(\bar{s}_j) \ge c + \alpha V_U$.
Substituting $V(\bar{s}_j)$ and performing the same algebraic steps as in Case(a) (replacing $p$ with $p_j$):
\[ w_j \ge \frac{c(1 - \alpha + \alpha p_j) + V_U \alpha(1 - \alpha)}{1 + \alpha p_j} \]
Let $RHS_j = \frac{c(1 - \alpha + \alpha p_j) + V_U \alpha(1 - \alpha)}{1 + \alpha p_j}$.

In this case, the condition for accepting an offer $w_j$ is $w_j \ge RHS_j$.
The critical observation is that $RHS_j$ is no longer a constant threshold.
Instead, it depends on $p_j$, which is specific to the offer $w_j$. Which implies the decision depends on both salary and job security.

The worker may weigh the offered salary $w_j$ against its associated firing probability $p_j$.
A high salary $w_j$ with a high $p_j$ might be less attractive than a slightly lower salary $w_k$ with a very low $p_k$.
The optimal strategy is to accept an offer $w_j$
if its value $w_j + \alpha V(\bar{s}_j)$ (which incorporates $p_j$) is greater than or equal to the value of continuing to search, $c + \alpha V_U$.
% 标记最后一页用于总页数计算
\label{LastPage}

\end{document}

